\subsection{Identificadores}

Un \textbf{identificador} es una secuencia de caracteres con las que se pueden
nombrar componentes del lenguaje como variables, funciones y procedimientos. Los
caracteres que se utilizan en un \textbf{identificador} pueden ser letras minúsculas,
dígitos, o el carácter de subrayado '\_'. Como en muchos otros lenguajes, el primer
carácter de un identificador no puede ser un dígito.

\subsection{Palabras clave}

Las \textbf{palabras clave} son un tipo de identificador particular que se emplean en el
lenguaje como secuencia de caracteres reservadas. Esto quiere decir que, cuando se escribe
en \textbf{c3P} una palabra clave, sólo puede mantener un único propósito y significado.

A continuación, se muestra una lista de las palabras clave en \textbf{c3P}:

\#add, b, break, c, call, continue, def, else, endfor, endfunc, endif, endproc,
endswitch, endwhile, f32, f64, for, func, hide, i8, i16, i32, i64, if, proc,
readonly, ret, str, switch,  while

\subsection{Constantes}

Una \textbf{constante} es un valor numérico o de caracteres que se utilizan como valores
literales para definir el valor de las variables. En \textbf{c3P}, se permite asignar
variables a valores de un tipo de dato diferente, puesto que en esta asignación se
está realizando una transformación del dato conocida como \textbf{casting}, en la
que se convierte el valor de la constante que se asigna a la variable en la representación
más precisa posible que describe con el tipo de dato de la variable su valor anterior.

\subsubsection{Constantes enteras}

Las constantes enteras se codifican en \textbf{c3P} a partir de una secuencia de dígitos.
Como este tipo de constantes representan a los números enteros, el rango de valores posible
comprende a todos los enteros positivos y negativos.

\subsubsection{Constantes de carácter}

Las constantes de carácter comprenden a todos los caracteres imprimibles, excepto los
caracteres de escape típicos en la mayoría de lenguajes de programación. Cabe mencionar
que existe un tipo de datos especifico con el que definir una cadena de caracteres,
que se representa delimitado en "", mientras que definir un sólo carácter es  ''

\subsubsection{Constantes de números reales}

Las constantes de números reales son aquellas que representan a un número decimal o
en coma flotante. La codificación de este tipo de constate consta de una secuencia
de dígitos que conforman la parte entera del mundo, seguido de un punto decimal y
una secuencia de dígitos que corresponden a la parte fraccionaria. Es importante
mencionar que en \textbf{c3P} no se puede omitir en la declaración de la constante
la parte fraccionaria, aunque sea nula.

\subsubsection{Constantes de valores booleanos}

Las constantes de valores booleanos son aquellas que pueden tener uno de sus
dos estados posibles (verdadero o falso), y cumplen las reglas de la aritmética
de Bool. En \textbf{c3P}, los booleanos no pueden transformarse a datos enteros,
como ocurren en lenguajes tales como C.

\subsection{Operadores}

Un \textbf{operador} es una palabra especial que representa en el lenguaje una operación,
que puede ser aritmética o lógica, y que afecta a uno o dos operandos.

\subsection{Separadores}

Un \textbf{separador} es un carácter que se utiliza para separar las palabras especiales
declaradas en un programa. En \textbf{c3P}, hay separadores que se emplean en la indexación,
en la definición de un contenedor, en la declaración de una variable, entre otros. En la
siguiente tabla se muestra cada separador y su uso en el lenguaje.

\begin{table}[H]
    \centering
    \begin{tabular}{|c|p{10cm}|}
         \hline
         \textbf{Nombre} & \textbf{Funcionalidad}  \\
         \hline
         \multirow{3}{*}{\centering{( )}} &
         - Llamada y declaración de funciones/procedimientos \par
         - Declaración de condiciones lógicas en If, For, While, etc. \par
         - Bloque de agrupamiento de expresiones aritméticas \\
         \hline
         \multirow{2}{*}{\centering{[ ]}} &
         - Indexación de Contenedores y cadenas de caracteres \par
         - Declaración de Contenedores de Elementos \\
         \hline
         \multirow{1}{*}{\centering{\{ \}}}
         & - Declaración de los Elementos de un Contenedor \\
         \hline
         \multirow{1}{*}{\centering{.}}
         & - Separador de los números decimales \\
         \hline
         \multirow{1}{*}{\centering{:}}
         & - Definición y declaración de variables \\
         \hline
         \multirow{1}{*}{\centering{,}}
         & - Separar parámetros, valores de un contenedor \\
         \hline
         \multirow{1}{*}{\centering{SPACE}}
          & - Carácter de espacio, de tabulación y nueva línea \\
         \hline
    \end{tabular}
    \caption{Separadores disponibles}
    \label{tab:tab_separadores_disponibles}
\end{table}

\subsection{Comentarios}

Los comentarios en c3P se definen mediante «?»

\begin{verbatim}
? Esto es un comentario
x : i32 = 10
\end{verbatim}
