\subsection{Tipos de Datos Primitivos}
\subsubsection{Tipos Enteros}

En \textbf{c3P}, los tipos enteros se diferencian no sólo por su identificador (nombre del
tipo de entero), sino también por el rango de valores que pueden almacenar, que se miden
por bits y pueden alcanzar los 64 bits.

\begin{table}[H]
    \centering
    \begin{tabular}{|c|c|c|}
         \hline
         \textbf{Identificador} & \textbf{NºBits} & \textbf{Rango} \\
         \hline
         i8 & 8 & -128 a 127 \\
         \hline
         i16 & 16 & -32768 a 32767 \\
         \hline
         i32 & 32 & -214483648 a 2147483647 \\
         \hline
         i64 & 64 & -9223372036854775808 a 9223372036854775807 \\
         \hline
    \end{tabular}
    \caption{Tipos Enteros}
    \label{tab:tab_tipos_enteros}
\end{table}

\subsubsection{Tipos de Números Reales}

Al igual que los números enteros, los tipos de números reales se diferencian tanto
por su identificador, sino también por su rango de valores que pueden almacenar,
que pueden alcanzar hasta los 64 bits. Cabe destacar que, dependiendo del ordenador
que esté ejecutando el programa, los rangos de valores varían.

\begin{table}[H]
    \centering
    \begin{tabular}{|c|c|c|}
         \hline
         \textbf{Identificador} & \textbf{NºBits} & \textbf{Rango} \\
         \hline
         f32 & 32 & F32\_MIN a F32\_MAX \\
         \hline
         f64 & 64 & F64\_MIN a F64\_MAX \\
         \hline
    \end{tabular}
    \caption{Tipos Enteros}
    \label{tab:tab_tipos_decimales}
\end{table}

\subsubsection{Tipos de Carácter}

Los tipos de caracteres permiten representar uno o varios caracteres de dígitos,
letras, u otros símbolos, a excepción de los caracteres especiales.

\begin{itemize}
    \item \textbf{c}: puede contener un valor que cumple '([a-zA-Z0-9\+\-\_\&\%...]+)*'
\end{itemize}

\subsubsection{Tipos Lógicos}

Los tipos de datos lógicos se codifican mediante el identificador «b», y pueden tomar dos
valores posibles (verdadero codificado en «T», o falso representado en «F»).

\subsection{Contenedores}

Un \textbf{contenedor} es una estructura de datos en la que se almacena valores de un
mismo tipo. Estos elementos están organizados internamente a partir de los índices del
contenedor, que marcan cada una de los posiciones en las que se almacena su contenido.
En \textbf{c3P}, los contenedores son de tamaño fijo y permiten operaciones de indexación.

\subsubsection{Declaración de contenedores}

Un contenedor se declara especificando el tipo de datos que tendrá cada elementos,
el nombre de la variable, y el número de elementos que puede almacenar, tal y como
se puede observar en el siguiente ejemplo:

\begin{verbatim}
mi_array : i32[10]
\end{verbatim}

Respecto al tamaño de un contenedor, puede declararse tanto con una constante literal
entera positiva, como con una variable del mismo tipo. Cabe mencionar que se permiten
tamaños de hasta la unidad.

\subsubsection{Inicialización de contenedores}

Los elementos de un contenedor se pueden inicializar de forma individual por medio de
la indexación, o también a partir de la declaración entre llaves de valores literales,
donde cada elemento tendrá asociado la posición en la que se define. En este ejemplo
se puede observar la segunda inicialización explicada:

\begin{verbatim}
mi_array : i32[5] = { 0, 1, 2, 3, 4 }
\end{verbatim}

Es importante considerar que en la inicialización se tienen que especificar todos los
valores del contenedor de elementos, lo que hace esta funcionalidad poco útil en
los casos en los que el tamaño de elementos sea muy grandes.

\subsubsection{Acceso a los elementos de un Contenedor}

En \textbf{c3P}, los elementos de un contenedor se pueden acceder por medio de su índice y
de la indexación. Para ello, se especifica el nombre del contenedor, seguido del índice del
elemento al que se quiere acceder, encerrado entre corchetes. A continuación se muestra un
ejemplo de esta funcionalidad:

\begin{verbatim}
mi_array[0] = 5
\end{verbatim}

\subsection{Tamaño de Contenedores y Cadena de caracteres}

En \textbf{c3P} para obtener el tamaño de un contenedor existe la función «arrlen».

\begin{verbatim}
mi_array : i32[3] = {0, 1, 2}
x : i32 = call arrlen mi_array
\end{verbatim}
