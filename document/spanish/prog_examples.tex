\subsection{hello\_world.c3p}

\begin{verbatim}
#add <c3p.base>

proc main()
    call showln "Hello, world!"
endproc    
\end{verbatim}

\subsection{mean\_age.c3p}
\begin{verbatim}
#add <c3p.base>

proc main()
    i32 : total_ages
    i16 : age_mean
    
    total_ages = call input "Inserta el número de edades ", "%i"
    
    for (i : i32 = 0, i < total_ages, i = i + 1)
        i32 : age = call input "Inserta la edad " + str(i), "%i"
        age_mean = age_mean + age
    endfor
    
    age_mean = age_mean / total_ages
    call showln "La edad media es: + str(i)
endproc   
\end{verbatim}

\subsection{fibonacci.c3p}

\begin{verbatim}
#add <c3p.base>

func fibonacci(n : i32)
    if (n > 1)
        f1 : i32 = call fibonacci n - 1
        f2 : i32 = call fibonacci n - 2
        ret f1 + f2
    endif
    
    ret n
endfunc

proc main()
    n10 : i32 = call fibonacci 10
    n20 : i32 = call fibonacci 20
    n30 : i32 = call fibonacci 30
    n40 : i32 = call fibonacci 40
    call showln "fib(10) = " + str(n10) 
    call showln "fib(20) = " + str(n20) 
    call showln "fib(30) = " + str(n30) 
    call showln "fib(40) = " + str(n40) 
endproc
\end{verbatim}