\subsection{hello\_world.c3p}

\begin{verbatim}
#add <c3p/base>

proc main()
    call showln "Hello, world!"
endproc
\end{verbatim}

\subsection{mean\_age.c3p}
\begin{verbatim}
#add <c3p/base>

proc main()
    i32 : total_ages
    i16 : age_mean

    total_ages = call input "Inserta el número de edades ", "%i"

    for (i : i32 = 0, i < total_ages, i = i + 1)
	stri : str = call str i
        i32 : age = call input "Inserta la edad " + stri, "%i"
        age_mean = age_mean + age
    endfor

    age_mean = age_mean / total_ages
    age_str_mean = call str age_mean
    call showln "La edad media es: + age_str_mean
endproc
\end{verbatim}

\subsection{fibonacci.c3p}

\begin{verbatim}
#add <c3p/base>

func fibonacci(n : i32)
    if (n > 1)
        f1 : i32 = call fibonacci n - 1
        f2 : i32 = call fibonacci n - 2
        ret f1 + f2
    endif

    ret n
endfunc

proc main()
    n10 : i32 = call fibonacci 10
    n20 : i32 = call fibonacci 20
    n30 : i32 = call fibonacci 30
    n40 : i32 = call fibonacci 40
    n10_str : str = call str n10
    n20_str : str = call str n20
    n30_str : str = call str n30
    n40_str : str = call str n40
    call showln "fib(10) = " + n10_str
    call showln "fib(20) = " + n20_str
    call showln "fib(30) = " + n30_str
    call showln "fib(40) = " + n40_str
endproc
\end{verbatim}
