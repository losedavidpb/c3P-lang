
Un programa en \textbf{c3P} puede existir dentro de uno o varios archivo fuente, y siempre uno
de ellos debe contener el procedimiento main, que es en el que se inicia la ejecución.
Cada fichero escrito en \textbf{c3P} tiene que tener la extensión .c3P

\subsection{Alcance de los datos}

Una variable declarada puede ser accesible sólo dentro de una función o procedimiento,
o dentro de un fichero. Las declaraciones realizadas fuera de funciones o procedimientos
son accesibles para todo el archivo y para otro ficheros siempre que no tenga el modificador
de acceso hide. Respecto a las variables locales, éstas son visibles dentro del cuerpo de
una función pero no fuera de ésta. Además de todo esto, hay que considerar que una declaración
en \textbf{c3P} no es visible por las declaraciones que la preceden, y las funciones o procedimientos
que no sean privadas son accesibles desde fuera del fichero siempre que éste se importe con \#add.