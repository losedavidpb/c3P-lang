
Un programa en \textbf{c3P} puede existir en un sólo fichero que tiene que tener
obligatoriamente el método main. Cada fichero escrito en \textbf{c3P} tiene que
tener la extensión .c3P o .c3p

\subsection{Alcance de los datos}

Una variable declarada puede ser accesible sólo dentro de una función o procedimiento,
o dentro de un fichero. Las declaraciones realizadas fuera de funciones o procedimientos
son accesibles para todo el archivo. Respecto a las variables locales, éstas son visibles dentro
del cuerpo de una función pero no fuera de ésta. Además de todo esto, hay que considerar que una
declaración en \textbf{c3P} no es visible por las declaraciones que la preceden.
