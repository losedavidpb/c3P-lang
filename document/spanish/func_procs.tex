\subsection{Declaración de Funciones y Procedimiento}

Por un lado, una función se declara en \textbf{c3P} mediante la palabra clave func,
seguido de su nombre, del tipo de dato que devuelve, y encerrado entre paréntesis,
de una lista de parámetros separados por comas. Además de todas estas componentes,
toda función en \textbf{c3P} tiene que terminar con la palabra clave ret, que es
en la que se indica el valor que devuelve.

\begin{verbatim}
func nombre : retorno (parámetros)
    ...
endfunc
\end{verbatim}

Por otro lado, un procedimiento se declara en \textbf{c3P} mediante la palabra clave proc,
seguido de su nombre, y encerrado entre paréntesis, de una lista de parámetros separados
por comas. A diferencia de las funciones, los procedimientos nunca incluyen la palabra
clave ret, porque no devuelven ningún valor. Además, ret no se puede usar para terminar
la ejecución de un procedimiento.

\begin{verbatim}
proc nombre (parámetros)
    ...
endproc
\end{verbatim}

El nombre puede ser cualquier identificador válido.

Los parámetros consta de cero o más parámetros, separados por comas. Un parámetro  consiste en un nombre y un tipo de datos para el parámetro. Estos parámetros son opcionales

\subsection{Definición de Funciones y Procedimiento}

La definición de una función o un procedimiento se escribe para especificar lo que realmente hace una función o un procedimiento. Una definición de función consiste en información sobre el nombre de la función, el tipo de retorno únicamente en las funciones y los tipos y nombres de los parámetros, junto con el cuerpo de la función.

\begin{verbatim}
func add_values : i64 (x : i32, y : i32)
  ret x + y
endfunc
\end{verbatim}

\begin{verbatim}
proc showSum (x : i32, y : i32)
  xy_str = call str (x + y)
  call show xy_str
endproc
\end{verbatim}

\subsection{Llamadas a Funciones y Procedimiento}

Para llamar a una función o procedimiento en \textbf{c3P}, se utiliza la palabra clave,
seguido del nombre de la función, y si es necesario, de los parámetros necesarios, separados
por comas.

\begin{verbatim}
x : i64 = call add_values 5, 3
call fibonacci 10
\end{verbatim}

\subsection{Parámetros de Funciones y Procedimiento}

En las funciones y procedimientos, los parámetros pueden ser cualquier expresión, como
un valor literal, un valor almacenado en una variable, o una expresión más compleja
construida mediante la combinación de estos.

En el cuerpo de la función o procedimiento, el parámetro se pasa por valor, lo que
significa que no se puede cambiar el valor pasado cambiando la copia local.

\begin{verbatim}
x : i32 = 23
call y1 x
call y2 x

func y1 : i32 (a : i32)
  a = 2 * a
  ret a
endfunc

proc y2 (a : i32)
  call show a
endproc
\end{verbatim}

\subsection{Función Principal}

Cualquier ejecución de un programa en \textbf{c3P} tiene que empezar siempre en
un procedimiento llamado «main» que no tiene parámetros.

\begin{verbatim}
proc main ()
  call show "Hello world"
endproc
\end{verbatim}

\subsection{Funciones Recursivas}

En \textbf{c3P} es posible definir funciones o procedimientos recursivos,
es decir, que en su ejecución se llamen a sí mismos, con en el siguiente ejemplo.

\begin{verbatim}
func fibonacci(n : i32)
    if (n > 1)
        f1 : i32 = call fibonacci n - 1
        f2 : i32 = call fibonacci n - 2
        ret f1 + f2
    endif

    ret n
endfunc
\end{verbatim}
