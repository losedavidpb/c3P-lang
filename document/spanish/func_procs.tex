\subsection{Declaración de Funciones y Procedimiento}

Por un lado, una función se declara en \textbf{c3P} mediante la palabra clave func,
seguido de su nombre, del tipo de dato que devuelve y, encerrado entre paréntesis,
una lista de parámetros separados por comas. Además de todas estas componentes,
toda función en \textbf{c3P} tiene que terminar con la palabra clave ret, seguido
de una expresión compleja o que no se defina con una llamada. Esto significa que
la última instrucción que tiene que definirse es la del retorno, y no puede
especificarse en cualquier otro momento, sino sólo al final. Cabe mencionar que
las funciones se cierran con la palabra clave endfunc.

\begin{verbatim}
func nombre : retorno (parámetros)
    ...
    ret valor_retorno
endfunc
\end{verbatim}

Por otro lado, un procedimiento se declara en \textbf{c3P} mediante la palabra clave proc,
seguido de su nombre, y encerrado entre paréntesis, de una lista de parámetros separados
por comas, y se cierran con la palabra clave endproc. A diferencia de las funciones, los
procedimientos nunca incluyen la palabra clave ret, porque no devuelven ningún valor. Además,
ret no se puede usar para terminar la ejecución de un procedimiento.

\begin{verbatim}
proc nombre (parámetros)
    ...
endproc
\end{verbatim}

El nombre puede ser cualquier identificador válido.
Los parámetros consta de cero o más parámetros, separados por comas. Un parámetro consiste en un
nombre y un tipo de datos para el parámetro.

\subsection{Definición de Funciones y Procedimiento}

La definición de una función o un procedimiento se escribe para especificar lo que realmente hace una
función o un procedimiento. Una definición de función consiste en información sobre el nombre de la función,
el tipo de retorno únicamente en las funciones y los tipos y nombres de los parámetros, junto con el cuerpo de la función.

\begin{verbatim}
func add_values : i64 (x : i32, y : i32)
  ret x + y
endfunc
\end{verbatim}

\begin{verbatim}
proc showSum (x : i32, y : i32)
  call show x + y
endproc
\end{verbatim}

\subsection{Llamadas a Funciones y Procedimiento}

Para llamar a una función o procedimiento en \textbf{c3P}, se utiliza la palabra clave <<call>>,
seguido del nombre de la función, y si es necesario, de los parámetros asociados, separados
por comas. A continuación se muestra un ejemplo de uso:

\begin{verbatim}
x : i64 = call add_values 5, 3
x = call fibonacci 10
\end{verbatim}

Cabe aclarar que si se declara la llamada a una función ésta siempre ha de estar asignada a una
variable local, global, o parámetro, a diferencia de los procedimientos, en la que se restringe
el caso opuesto.

\subsection{Parámetros de Funciones y Procedimiento}

En las funciones y procedimientos, los parámetros pueden ser cualquier expresión, como
un valor literal, un valor almacenado en una variable, o una expresión más compleja
construida mediante la combinación de estos, a excepción de las llamadas a funciones.

En el cuerpo de la función o procedimiento, el parámetro se pasa por valor, lo que
significa que no se puede cambiar el valor pasado cambiando la copia local.

\begin{verbatim}
x : i32 = 23
x = call y1 x
call y2 x

func y1 : i32 (a : i32)
  a = 2 * a
  ret a
endfunc

proc y2 (a : i32)
  call show a
endproc
\end{verbatim}

\subsection{Función Principal}

Cualquier ejecución de un programa en \textbf{c3P} tiene que empezar siempre en
un procedimiento llamado «main» que nunca tiene parámetros.

\begin{verbatim}
proc main ()
  call show "Hello world"
endproc
\end{verbatim}

\subsection{Funciones Recursivas}

En \textbf{c3P} es posible definir funciones o procedimientos recursivos,
es decir, que en su ejecución se llamen a sí mismos, como el siguiente ejemplo.

\begin{verbatim}
func fibonacci : i32(n : i32)
    res : i32 = n

    if (n > 1)
        f1 : i32 = call fibonacci n - 1
        f2 : i32 = call fibonacci n - 2
        res = f1 + f2
    endif

    ret res
endfunc
\end{verbatim}
