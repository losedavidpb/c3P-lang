\subsection{Expresiones}

En \textbf{c3P}, una \textbf{expresión} es un conjunto de uno o más operandos y operadores, mientras
que un \textbf{operando} son objetos descritos como constantes, variables, y llamadas a funciones que
devuelven valores de un tipo determinado. A continuación, se explica cada uno de los operadores:

\subsection{Operador de Asignación}

Los operadores de asignación se utilizan para almacenan valores en variables, y se codifica
en \textbf{c3P} mediante el símbolo «=». Como ya se mencionó anteriormente, si el valor
derecho que se asigna a la variable presenta un tipo de dato diferente, en \textbf{c3P} se
hará una operación de \textbf{casting}, siempre que la transformación sea posible. En el
siguiente ejemplo, se muestra el operador de asignación, y una situación de \textbf{casting},
tanto una válida como otra que provoca un error de ejecución.

\begin{verbatim}
x1 : i32 = 20
y1 : i64 = 10.3

? Casting válido porque ambos datos son numéricos
? y puede transformarse porque puede guardarse la
? parte entera del valor tipo i64 (x1 = 10)
x1 = y1

x2 : i32 = 20
y2 : i64 = 10.3

? Casting válido porque ambos datos son numéricos
? y puede transformarse porque puede guardarse la
? parte entera como decimal (y2 = 20.0)
y2 = x2

carac : c = 'H'
num : i32 = 10

? ¡¡ERROR!! El casting sólo se permite en
? datos numéricos o si ambos valores son caracteres
num = carac
\end{verbatim}

\subsection{Operadores Aritméticos}

\textbf{c3P} proporciona operadores aritméticos como el de suma, resta, multiplicación, y división.
A continuación, se muestra ejemplos de estos operadores:

\begin{verbatim}
x : i32 = 5 + 3
y : f32 = 10.23 + 37.332
z : f32:  = x + y

x : i32 = 5 - 3
y : f32 = 57.223 - 10.903
z : f64 = x - y

x : i32 = 5 * 3
y : f32 = 47.4 * 1.001
z : f32 = x * y

x : f32 = 5 / 3
y : f64 = 940.0 / 20.2
z : f64 = x / y

x : f32 = 5 % 3
y : f64 = 940.0 % 20.2
z : f64 = x % y

x : f32 = 5 ^ 3
y : f64 = 940.0 ^ 20.2
z : f64 = x ^ y
\end{verbatim}

\subsection{Operadores de Comparación}

Los operadores de comparación se emplean para comprobar si se satisface una condición
en la que participan dos operadores que se relacionan. El resultado de este operador
siempre es booleano. A continuación, se muestra cada uno de los operadores de comparación
que existen en \textbf{c3P}.

El operador de igualdad «==» comprueba la igualdad de dos operandos:

\begin{verbatim}
if (x == y)
  call show ("x es igual a y")
else
  call show ("x no es igual a y")
endif
\end{verbatim}

El operador de desigualdad «!=» comprueba la desigualdad de sus dos operandos.

\begin{verbatim}
if (x != y)
  call show ("x no es igual a y")
else
  call show ("x es igual a y")
endif
\end{verbatim}

Además de la igualdad y la desigualdad, existen una serie de operadores con los que
comprobar si un valor es menor que, mayor que, menor o igual que, o mayor o igual
que otro operando.

\begin{verbatim}
if (x < y)
  call show ("x es menor que y")
endif
\end{verbatim}

\begin{verbatim}
if (x <= y)
  call show ("x es menor o igual que y")
endif
\end{verbatim}

\begin{verbatim}
if (x > y)
  call show ("x es mayor que y")
endif
\end{verbatim}

\begin{verbatim}
if (x >= y)
  call show ("x es mayor o igual que y")
endif
\end{verbatim}

\subsection{Operadores Lógicos}

Los operadores lógicos se utilizan para comprobar la veracidad de dos
condiciones expresadas mediante valores booleanos. A continuación, se
explica cada uno de los proporcionados en \textbf{c3P}:

\begin{itemize}
    \item El operador \textbf{and} comprueba si dos expresiones son ambas verdaderas.
\end{itemize}

\begin{verbatim}
if ((x == 5) and (y == 10))
  call show ("x es 5 e y es 10")
endif
\end{verbatim}

\begin{itemize}
    \item El operador \textbf{or} comprueba si al menos una de las dos expresiones es verdadera.
\end{itemize}

\begin{verbatim}
if ((x == 5) or (y == 10))
   call showln ("x es 5 o y es 10")
endif
\end{verbatim}

\begin{itemize}
    \item Puede anteponer a una expresión lógica un operador de negación \textbf{not} para invertir el valor de la expresión booleana:
\end{itemize}

\begin{verbatim}
if (not (x == 5))
  call showln ("x no es 5")
endif
\end{verbatim}

\subsection{Llamadas a Funciones como Expresiones}

Para facilitar la definición de una expresión, en \textbf{c3P} se considera a
cualquier función una función, por lo que se pueden asignar la llamada de una
de estas a una variable.

\begin{verbatim}
func y : i32(x : i32)
    ret x * 2
endfunc

a : i32 = 10 + (call function 20);
\end{verbatim}

\subsection{Precedencia de Operadores}

Cuando una expresión contiene varios operadores, como a + b * f(), los operadores se agrupan según las reglas de precedencia.

A continuación se presenta una lista de tipos de expresiones, presentadas primero en orden de mayor precedencia. A veces, dos o más operadores tienen la misma precedencia; todos esos operadores se aplican de izquierda a derecha a menos que se indique lo contrario.

\begin{enumerate}
  \item Llamadas a funciones.
  \item Operadores unarios, incluyendo la negación lógica..
  \item Expresiones de multiplicación, división, división modular y exponentes.
  \item Expresiones de suma y resta.
  \item Expresiones de mayor que, menor que, mayor o igual que, y menor o igual que.
  \item Expresiones de igualdad y desigualdad.
  \item Expresiones lógicas AND.
  \item Expresiones lógicas OR.
  \item Todas las expresiones de asignación, incluida la asignación compuesta. \newline Cuando aparecen varias sentencias de asignación como subexpresiones en una única expresión mayor, se evalúan de derecha a izquierda.
  \item Expresiones con operadores de coma. 
\end{enumerate}

\subsection{Orden de Evaluación}

En c3P todo programa comienza en un procedimiento llamado obligatoriamente «main». 

No se puede asumir que las subexpresiones múltiples se evalúan en el orden que parece natural.