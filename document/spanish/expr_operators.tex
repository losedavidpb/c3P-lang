\subsection{Expresiones}

En \textbf{c3P}, una \textbf{expresión} es un conjunto de uno o más operandos y operadores, mientras
que un \textbf{operando} son objetos descritos como constantes, y variables. Cabe aclarar que se
puede definir una expresión simple con la llamada a una función siempre que se defina en la asignación
de una variable local o global, o en un elemento de un contenedor. A continuación, se explica cada uno
de los operadores existentes:

\subsection{Operador de Asignación}

Los operadores de asignación se utilizan para almacenan valores en variables, y se codifican
en \textbf{c3P} mediante el símbolo «=». En \textbf{c3P} siempre ha de tenerse un control
absoluto de los tipos de datos y rangos de valores que se devuelvan de una expresión, porque
no se admiten operaciones de \textbf{casting} y tampoco se aplica ninguna transformación
para que se admita hacer cálculos con tipos difernetes.

\subsection{Operadores Aritméticos}

\textbf{c3P} proporciona operadores aritméticos típicos en muchos lenguajes de programacion como
el de la suma, resta, multiplicación, y división. Sin embargo, \textbf{c3p} admite como operadores
el módulo y exponente, que no son tan frecuentes y comunes. A continuación, se muestra ejemplos de
estos operadores:

\begin{verbatim}
x : i32 = 5 + 3
y : f32 = 10.23 + 37.332
z : f32:  = x + y

x : i32 = 5 - 3
y : f32 = 57.223 - 10.903
z : f64 = x - y

x : i32 = 5 * 3
y : f32 = 47.4 * 1.001
z : f32 = x * y

x : f32 = 5 / 3
y : f64 = 940.0 / 20.2
z : f64 = x / y

x : f32 = 5 % 3
y : f64 = 940.0 % 20.2
z : f64 = x % y

x : f32 = 5 ^ 3
y : f64 = 940.0 ^ 20.2
z : f64 = x ^ y
\end{verbatim}

\subsection{Operadores de Comparación}

Los operadores de comparación se emplean para comprobar si se satisface una condición
en la que participan dos operadores que se relacionan. El resultado de este operador
siempre es booleano. A continuación, se muestra cada uno de los operadores de comparación
que existen en \textbf{c3P}.

El operador de igualdad «==» comprueba la igualdad de dos operandos:

\begin{verbatim}
if (x == y)
  call show 'T'
else
  call show 'F'
endif
\end{verbatim}

El operador de desigualdad «!=» comprueba la desigualdad de sus dos operandos.

\begin{verbatim}
if (x != y)
  call show 'T'
else
  call show 'F'
endif
\end{verbatim}

Además de la igualdad y la desigualdad, existen una serie de operadores con los que
comprobar si un valor es menor que, mayor que, menor o igual que, o mayor o igual
que otro operando.

\begin{verbatim}
if (x < y)
  call show 'T'
endif
\end{verbatim}

\begin{verbatim}
if (x <= y)
  call show 'T'
endif
\end{verbatim}

\begin{verbatim}
if (x > y)
  call show 'T'
endif
\end{verbatim}

\begin{verbatim}
if (x >= y)
  call show 'T'
endif
\end{verbatim}

\subsection{Operadores Lógicos}

Los operadores lógicos se utilizan para comprobar la veracidad de dos
condiciones expresadas mediante valores booleanos. A continuación, se
explican cada uno de los proporcionados en \textbf{c3P}:

\begin{itemize}
    \item El operador \textbf{and} comprueba si dos expresiones son verdaderas.
\end{itemize}

\begin{verbatim}
if ((x == 5) and (y == 10))
  call show 'T'
endif
\end{verbatim}

\begin{itemize}
    \item El operador \textbf{or} comprueba si al menos una de las dos expresiones es verdadera.
\end{itemize}

\begin{verbatim}
if ((x == 5) or (y == 10))
   call showln 'T'
endif
\end{verbatim}

\begin{itemize}
    \item Se puede anteponer a una expresión lógica un operador de negación \textbf{not}
    para invertir el valor de la expresión booleana:
\end{itemize}

\begin{verbatim}
if (not (x == 5))
  call showln 'T'
endif
\end{verbatim}

\subsection{Llamadas a Funciones como Expresiones}

Tal y como se mencionó anteriormente, las llamadas a funciones
sólo se admiten en expresiones simples en las que no existe
ninguna operación, como ocurre en el siguiente ejemplo:

\begin{verbatim}
func y : i32(x : i32)
    ret x * 2
endfunc

num1 : i32 = 10
num2 : i32 = call y 20
res : i32 = a + c
call showln res
\end{verbatim}

\subsection{Precedencia de Operadores}

Cuando una expresión contiene varios operadores, como a + b * 2, los operadores se agrupan según las reglas de precedencia.

A continuación se presenta una lista de tipos de expresiones, presentadas primero en orden de mayor precedencia. A veces, dos o
más operadores tienen la misma precedencia; todos estos operadores se aplican de izquierda a derecha a menos que se indique lo contrario.

\begin{enumerate}
  \item Negación lógica NOT
  \item Expresiones de multiplicación, división, división modular y exponentes.
  \item Expresiones de suma y resta.
  \item Expresiones de mayor que, menor que, mayor o igual que, y menor o igual que.
  \item Expresiones de igualdad y desigualdad.
  \item Expresiones lógicas AND.
  \item Expresiones lógicas OR.
  \item Todas las expresiones de asignación, incluida la asignación compuesta.
  \newline Cuando aparecen varias sentencias de asignación como subexpresiones en una única expresión mayor, se evalúan de derecha a izquierda.
  \item Expresiones con operadores de coma.
\end{enumerate}

\subsection{Orden de Evaluación}

En c3P todo programa comienza en un procedimiento llamado obligatoriamente «main».
No se puede asumir que las subexpresiones múltiples se evalúan en el orden que parece natural.
